Con el fin constatar el correcto funcionamiento del hardware diseñado en el Capítulo \ref{cap:sadq}, se realizó una prueba consistente en en simular los pulsos digitales a capturar mediante un módulo auxiliar diseñado especialmente para este propósito, sumado a las herramientas disponibles en el software de desarrollo Vivado: VIO (Virtual Input/Output)\cite{XilinxVirtualSuite} e ILA (Integrated Logic Analyzer)\cite{XilinxIntegratedPG172}, contrastando así la integridad y duración de los pulsos capturados respecto a los datos recibidos mediante comunicación serial hacia el computador.

%La segunda prueba consiste en simular pulsos analógicos de carga conocida mediante un generador de púlsos y un pequeño circuito, logrando emitir pulsos con características muy similares a las señales de detección provenientes de un sTGC. Los pulsos simulados serán recibidos por la interfaz ASD, la que estará interconectada mediante sus puertos LVDS a la tarjeta de desarrollo Trenz en la que se implementó el sistema de adquisicón. El objetivo de esta prueba será contrastar los pulsos recibidos por comunicación serial con la carga del pulso de entrada ingresado en la interfaz ASD.
%
%\section{Prueba de Hardware Mediante Pulsos Digitales}


El módulo auxiliar de simulación tiene como objetivo generar un exhaustivo barrido de señales abarcando todos los puertos de adquisición y simula señales equivalentes a 36 eventos de detección en diferentes vértices de interacción de un detector sTGC imaginario. Por ejemplo, la Figura \ref{img:digital_test_example} representa uno de los 36 eventos simulados, particularmente el evento 10, en donde se observan 6 canales coloreados en naranjo, tres por cada eje coordenado. Coloreado en celeste se encuentra el vértice de interacción estimado, en el centro la intersección de canales coloreada en verde. Cada evento representa un vértice de adquisición diferente y excita siempre 3 puertos de adquisición por cada eje coordenado, lo que se traduce 36 diferentes combinaciones de señales para la representación de eventos. Además, la duración de las señales es diferente en cada evento, partiendo con 36 ciclos de reloj de duración para las señales del primer evento y terminando con 1 ciclo de reloj de duración para las señales del último evento enviado, permitiendo poner a prueba la resolución temporal del sistema de adquisición.
	
	\begin{figure}[h]
		\centering
		\includegraphics[scale=0.65]{digital_test_example.png}
		\caption{Ejemplo de uno de los 36 eventos, correspondiente al evento de prueba número 10. }
		\label{img:digital_test_example}
	\end{figure}
	
	El envío de los 36 eventos se inició mediante un botón virtual configurado en un bloque VIO, mientras las señales internas se monitorearon con un bloque ILA. Los datos generados por el sistema de adquisición diseñado fueron leidos a través de una consola serial incluida en la interfaz del software Xilinx SDK. Las Figuras \ref{img:e1} y \ref{img:e4} corresponden a capturaa de pantalla de la interfaz ILA e ilustran las señales internas asociadas a los eventos 1 y 4 respectivamente. En ambas Figuras se incluyen 6 señales: \textit{Start\_bttn}, correspondiente a la señal emitida por el botón configurado en la VIO; \textit{Input\_Event}, vector de 15bits que representa los datos enviados desde el módulo auxiliar hacia el sistema de adquisición; \textit{Trigger}, correspondiente a la simulación de una señal de disparo con un delay de 48 ciclos de reloj; la señal para habilitar la escritura en la memoria FIFO, nombrada como ]\textit{FIFO\_write\_enable};\textit{FIFO\_data\_input}, vector de 64bits asociado a los datos recibidos por el sistema de adquisición listos para ser guardados en la memoria FIFO; y \textit{FIFO\_empty\_flag}, la cual corresponde a la señal emitida por la memoria FIFO cuando está vacía.
	
	En la Figura \ref{img:e1} se ilustran las señales internas asociadas al primer evento enviado y se situan en ellas 4 marcadores azules y un marcador verde. Los primeros dos marcadores azules corresponden al inicio y término del envío de señales a través del vector \textit{Input\_event}, indicando que la duración de este grupo de señales es de 9 ciclos de reloj. Dado que la frecuencia de reloj utilizada para el bloque ILA es de 100MHz, pero el módulo auxiliar opera a 400MHz, se tiene entonces que la duración de las señales emitidas correspondientes al primer evento es 4 veces la cantidad de ciclos demarcada, significando una duración de 36 ciclos (90ns) para este evento. Así se comprueba que la señal ha sido correctamente emitida. Además su valor en notación hexadecimal corresponde a \textit{0x07e0}, que se traduce en binario a \textit{0b0000011111100000} y representa que los canales excitados corresponden a los primeros 3 de cada eje coordenado
	
	Los últimos dos marcadores azules de la Figura \ref{img:e1} demarcan el inicio y término de la escritura en memoria (\textit{FIFO\_write\_enable}) de los datos capturados por el sistema de adquisición. Este proceso de escritura toma 16 ciclos de reloj, que efectivamente corresponden a los 16 ciclos necesarios para escribir la información de cada uno de los 16 canales capturados por el sistema de adquisición según el reloj de 100MHz asociado a la memoria FIFO. El marcador verde está situado en medio de la información a a ser escrita en la memoria FIFO indicada por el vector de 64 bits \textit{FIFO\_data\_input}, donde se puede observar el número hexadecimal \textit{0x1FFFFFFFFE} que representa la duración de los 6 canales intermedios excitados por este evento. Sumando el número de bits en alto que contiene este dato hexadecimal se obtiene 36, lo que corresponde a la los ciclos de reloj que dura el evento 1.
	
	La señal \textit{trigger} indicada en la Figura \ref{img:e1} no es observable debido a la frecuencia de reloj el analizador lógico, el cual es 4 veces más lento que el reloj al que opera la emisión y captura de pulsos de disparo. En la \ref{img:e4} sí es posible observar la señal de disparo ya que corresponde al cuarto evento envíado por el módulo auxilar. En esta Figura se observa una señal de disparo correspondiente a 12 ciclos de reloj del analizador lógico, que se traducen a 48 ciclos de reloj en el dominio de la adquisición de datos y representa a un delay de 120ns respecto a la emisión del evento representado en la Figura \ref{img:e4}.
	
	\begin{figure}[ht]
		\centering
		\includegraphics[scale=0.7]{e1.png}
		\caption{Captura de pantalla de la interfaz ILA, donde se ilustra la recepción del primer evento de prueba.}
		\label{img:e1}
	\end{figure}
	
	
	\begin{figure}[ht]
		\centering
		\includegraphics[scale=0.7]{e4.png}
		\caption{Ejemplo de uno de los 36 eventos, correspondiente al evento de prueba número 4. }
		\label{img:e4}
	\end{figure}

	Finalmente, se realizó este experimento 28 veces para sumar un total de 1008 eventos capturados con la misma secuencia y duración de señales. Los eventos fueron recepcionados mediante comunicación serial y analizados a través de un programa para contar los bits contenidos en cada canal de cada evento. El conteo de bits se realizó con el algoritmo de Brian Kernighan\cite{SinghCountC++} para conteo de bits en números enteros. Así se comprobó que para los 1008 eventos se logró capturar el 100\% de las señales emitidas y se pudo muestrear correctamente el 100\% de la duración de cada señal con una resolución temporal de 2,5ns. El programa utilizado y los archivos generados en la adquisición de eventos se encuentran disponibles en el repositorio de este proyecto. La Tabla \ref{tab:datos-serial} corresponde a los datos obtenidos por comunicación serial ya procesados mediante el algoritmo para conteo de bits y representa la duración de cada evento en ciclos de reloj, considerando un reloj de 400MHz.	
	
	
	\begin{table}[ht]
		\centering
		\resizebox{\textwidth}{!}{%
			\begin{tabular}{|l|r|r|r|r|r|r|r|r|r|r|r|r|r|r|r|r|}
				\hline
				\rowcolor[HTML]{70AD47} 
				{\color[HTML]{FFFFFF} \textbf{Evento}} &
				\multicolumn{1}{l|}{\cellcolor[HTML]{70AD47}{\color[HTML]{FFFFFF} \textbf{Ch0}}} &
				\multicolumn{1}{l|}{\cellcolor[HTML]{70AD47}{\color[HTML]{FFFFFF} \textbf{Ch1}}} &
				\multicolumn{1}{l|}{\cellcolor[HTML]{70AD47}{\color[HTML]{FFFFFF} \textbf{Ch2}}} &
				\multicolumn{1}{l|}{\cellcolor[HTML]{70AD47}{\color[HTML]{FFFFFF} \textbf{Ch3}}} &
				\multicolumn{1}{l|}{\cellcolor[HTML]{70AD47}{\color[HTML]{FFFFFF} \textbf{Ch4}}} &
				\multicolumn{1}{l|}{\cellcolor[HTML]{70AD47}{\color[HTML]{FFFFFF} \textbf{Ch5}}} &
				\multicolumn{1}{l|}{\cellcolor[HTML]{70AD47}{\color[HTML]{FFFFFF} \textbf{Ch6}}} &
				\multicolumn{1}{l|}{\cellcolor[HTML]{70AD47}{\color[HTML]{FFFFFF} \textbf{Ch7}}} &
				\multicolumn{1}{l|}{\cellcolor[HTML]{70AD47}{\color[HTML]{FFFFFF} \textbf{Ch8}}} &
				\multicolumn{1}{l|}{\cellcolor[HTML]{70AD47}{\color[HTML]{FFFFFF} \textbf{Ch9}}} &
				\multicolumn{1}{l|}{\cellcolor[HTML]{70AD47}{\color[HTML]{FFFFFF} \textbf{Ch10}}} &
				\multicolumn{1}{l|}{\cellcolor[HTML]{70AD47}{\color[HTML]{FFFFFF} \textbf{Ch11}}} &
				\multicolumn{1}{l|}{\cellcolor[HTML]{70AD47}{\color[HTML]{FFFFFF} \textbf{Ch12}}} &
				\multicolumn{1}{l|}{\cellcolor[HTML]{70AD47}{\color[HTML]{FFFFFF} \textbf{Ch13}}} &
				\multicolumn{1}{l|}{\cellcolor[HTML]{70AD47}{\color[HTML]{FFFFFF} \textbf{Ch14}}} &
				\multicolumn{1}{l|}{\cellcolor[HTML]{70AD47}{\color[HTML]{FFFFFF} \textbf{Ch15}}} \\ \hline
				\rowcolor[HTML]{E2EFDA} 
				\textbf{1}  & 0  & 0  & 0  & 0  & 0  & 36 & 36 & 36 & 36 & 36 & 36 & 0  & 0  & 0  & 0  & 0 \\ \hline
				\rowcolor[HTML]{FFFFFF} 
				\textbf{2}  & 0  & 0  & 0  & 0  & 35 & 35 & 35 & 0  & 35 & 35 & 35 & 0  & 0  & 0  & 0  & 0 \\ \hline
				\rowcolor[HTML]{E2EFDA} 
				\textbf{3}  & 0  & 0  & 0  & 34 & 34 & 34 & 0  & 0  & 34 & 34 & 34 & 0  & 0  & 0  & 0  & 0 \\ \hline
				\rowcolor[HTML]{FFFFFF} 
				\textbf{4}  & 0  & 0  & 33 & 33 & 33 & 0  & 0  & 0  & 33 & 33 & 33 & 0  & 0  & 0  & 0  & 0 \\ \hline
				\rowcolor[HTML]{E2EFDA} 
				\textbf{5}  & 0  & 32 & 32 & 32 & 0  & 0  & 0  & 0  & 32 & 32 & 32 & 0  & 0  & 0  & 0  & 0 \\ \hline
				\rowcolor[HTML]{FFFFFF} 
				\textbf{6}  & 31 & 31 & 31 & 0  & 0  & 0  & 0  & 0  & 31 & 31 & 31 & 0  & 0  & 0  & 0  & 0 \\ \hline
				\rowcolor[HTML]{E2EFDA} 
				\textbf{7}  & 0  & 0  & 0  & 0  & 0  & 30 & 30 & 30 & 0  & 30 & 30 & 30 & 0  & 0  & 0  & 0 \\ \hline
				\rowcolor[HTML]{FFFFFF} 
				\textbf{8}  & 0  & 0  & 0  & 0  & 29 & 29 & 29 & 0  & 0  & 29 & 29 & 29 & 0  & 0  & 0  & 0 \\ \hline
				\rowcolor[HTML]{E2EFDA} 
				\textbf{9}  & 0  & 0  & 0  & 28 & 28 & 28 & 0  & 0  & 0  & 28 & 28 & 28 & 0  & 0  & 0  & 0 \\ \hline
				\rowcolor[HTML]{FFFFFF} 
				\textbf{10} & 0  & 0  & 27 & 27 & 27 & 0  & 0  & 0  & 0  & 27 & 27 & 27 & 0  & 0  & 0  & 0 \\ \hline
				\rowcolor[HTML]{E2EFDA} 
				\textbf{11} & 0  & 26 & 26 & 26 & 0  & 0  & 0  & 0  & 0  & 26 & 26 & 26 & 0  & 0  & 0  & 0 \\ \hline
				\rowcolor[HTML]{FFFFFF} 
				\textbf{12} & 25 & 25 & 25 & 0  & 0  & 0  & 0  & 0  & 0  & 25 & 25 & 25 & 0  & 0  & 0  & 0 \\ \hline
				\rowcolor[HTML]{E2EFDA} 
				\textbf{13} & 0  & 0  & 0  & 0  & 0  & 24 & 24 & 24 & 0  & 0  & 24 & 24 & 24 & 0  & 0  & 0 \\ \hline
				\rowcolor[HTML]{FFFFFF} 
				\textbf{14} & 0  & 0  & 0  & 0  & 23 & 23 & 23 & 0  & 0  & 0  & 23 & 23 & 23 & 0  & 0  & 0 \\ \hline
				\rowcolor[HTML]{E2EFDA} 
				\textbf{15} & 0  & 0  & 0  & 22 & 22 & 22 & 0  & 0  & 0  & 0  & 22 & 22 & 22 & 0  & 0  & 0 \\ \hline
				\rowcolor[HTML]{FFFFFF} 
				\textbf{16} & 0  & 0  & 21 & 21 & 21 & 0  & 0  & 0  & 0  & 0  & 21 & 21 & 21 & 0  & 0  & 0 \\ \hline
				\rowcolor[HTML]{E2EFDA} 
				\textbf{17} & 0  & 20 & 20 & 20 & 0  & 0  & 0  & 0  & 0  & 0  & 20 & 20 & 20 & 0  & 0  & 0 \\ \hline
				\rowcolor[HTML]{FFFFFF} 
				\textbf{18} & 19 & 19 & 19 & 0  & 0  & 0  & 0  & 0  & 0  & 0  & 19 & 19 & 19 & 0  & 0  & 0 \\ \hline
				\rowcolor[HTML]{E2EFDA} 
				\textbf{19} & 0  & 0  & 0  & 0  & 0  & 18 & 18 & 18 & 0  & 0  & 0  & 18 & 18 & 18 & 0  & 0 \\ \hline
				\rowcolor[HTML]{FFFFFF} 
				\textbf{20} & 0  & 0  & 0  & 0  & 17 & 17 & 17 & 0  & 0  & 0  & 0  & 17 & 17 & 17 & 0  & 0 \\ \hline
				\rowcolor[HTML]{E2EFDA} 
				\textbf{21} & 0  & 0  & 0  & 16 & 16 & 16 & 0  & 0  & 0  & 0  & 0  & 16 & 16 & 16 & 0  & 0 \\ \hline
				\rowcolor[HTML]{FFFFFF} 
				\textbf{22} & 0  & 0  & 15 & 15 & 15 & 0  & 0  & 0  & 0  & 0  & 0  & 15 & 15 & 15 & 0  & 0 \\ \hline
				\rowcolor[HTML]{E2EFDA} 
				\textbf{23} & 0  & 14 & 14 & 14 & 0  & 0  & 0  & 0  & 0  & 0  & 0  & 14 & 14 & 14 & 0  & 0 \\ \hline
				\rowcolor[HTML]{FFFFFF} 
				\textbf{24} & 13 & 13 & 13 & 0  & 0  & 0  & 0  & 0  & 0  & 0  & 0  & 13 & 13 & 13 & 0  & 0 \\ \hline
				\rowcolor[HTML]{E2EFDA} 
				\textbf{25} & 0  & 0  & 0  & 0  & 0  & 12 & 12 & 12 & 0  & 0  & 0  & 0  & 12 & 12 & 12 & 0 \\ \hline
				\rowcolor[HTML]{FFFFFF} 
				\textbf{26} & 0  & 0  & 0  & 0  & 11 & 11 & 11 & 0  & 0  & 0  & 0  & 0  & 11 & 11 & 11 & 0 \\ \hline
				\rowcolor[HTML]{E2EFDA} 
				\textbf{27} & 0  & 0  & 0  & 10 & 10 & 10 & 0  & 0  & 0  & 0  & 0  & 0  & 10 & 10 & 10 & 0 \\ \hline
				\rowcolor[HTML]{FFFFFF} 
				\textbf{28} & 0  & 0  & 9  & 9  & 9  & 0  & 0  & 0  & 0  & 0  & 0  & 0  & 9  & 9  & 9  & 0 \\ \hline
				\rowcolor[HTML]{E2EFDA} 
				\textbf{29} & 0  & 8  & 8  & 8  & 0  & 0  & 0  & 0  & 0  & 0  & 0  & 0  & 8  & 8  & 8  & 0 \\ \hline
				\rowcolor[HTML]{FFFFFF} 
				\textbf{30} & 7  & 7  & 7  & 0  & 0  & 0  & 0  & 0  & 0  & 0  & 0  & 0  & 7  & 7  & 7  & 0 \\ \hline
				\rowcolor[HTML]{E2EFDA} 
				\textbf{31} & 0  & 0  & 0  & 0  & 0  & 6  & 6  & 6  & 0  & 0  & 0  & 0  & 0  & 6  & 6  & 6 \\ \hline
				\rowcolor[HTML]{FFFFFF} 
				\textbf{32} & 0  & 0  & 0  & 0  & 5  & 5  & 5  & 0  & 0  & 0  & 0  & 0  & 0  & 5  & 5  & 5 \\ \hline
				\rowcolor[HTML]{E2EFDA} 
				\textbf{33} & 0  & 0  & 0  & 4  & 4  & 4  & 0  & 0  & 0  & 0  & 0  & 0  & 0  & 4  & 4  & 4 \\ \hline
				\rowcolor[HTML]{FFFFFF} 
				\textbf{34} & 0  & 0  & 3  & 3  & 3  & 0  & 0  & 0  & 0  & 0  & 0  & 0  & 0  & 3  & 3  & 3 \\ \hline
				\rowcolor[HTML]{E2EFDA} 
				\textbf{35} & 0  & 2  & 2  & 2  & 0  & 0  & 0  & 0  & 0  & 0  & 0  & 0  & 0  & 2  & 2  & 2 \\ \hline
				\rowcolor[HTML]{FFFFFF} 
				\textbf{36} & 1  & 1  & 1  & 0  & 0  & 0  & 0  & 0  & 0  & 0  & 0  & 0  & 0  & 1  & 1  & 1 \\ \hline
			\end{tabular}%
		}
		\caption{Ejemplo de tabla de datos recibida por comunicación serial, correspondiente al experimento número 15. Los números representan la duración de cada señal en ciclos de reloj de 400MHz.}
		\label{tab:datos-serial}
	\end{table}
%\section{Prueba de Hardware Mediante Pulsos Analógicos}
%
%explicacion
%
%objetivos
%
%disposicion experimental, recursos utilizados, metodo
%
%resultados
%
%conclusiones, datos, graficos
