Muons are subatomic particles originated by the interaction and decay of other particles. Muons originate mainly from cosmic radiation from outer space and penetrate the atmosphere, even reaching the earth's crust and passing through the matter they encounter on their way.Measuring the energy of a muon after passing through matter allows us to know the density of the materials traversed, so the detection of muons is a captivating study area for terrain and structure analysis.

The ``Data acquisition system for muon detectors'' was born as a requirement of the CCTVal (Centro Científico Tecnológico of Valparaíso) within the framework of the ``sTGC Minería'' project, whose objective is to carry out muonic tomography of mining land using sTGC detectors. A detector emits electrical signals representing the position and energy associated with the passage of a muon, which is why an acquisition system is required to capture these signals and deliver them to a subsequent analysis system for the detected muon characterization.

This dissertation describes a prototype acquisition system that fulfills the functions of sampling digital signals from a detector, discriminating the detection accuracy by reading an external trigger signal, and sending the captured information to an external computer to store and process the acquired data. The acquisition system must be capable of sampling 16 digital signals on the order of nanoseconds and must be designed with replication and scaling in mind to facilitate the connection of additional detectors. The designing process of this system sets a valuable precedent for CCTVal, so the development process and the insights gained are jointly documented in this report and the associated Git repository.

\textbf {Keywords:}  sTGC Detectors,  Muons, FPGA, Data Acquisition.